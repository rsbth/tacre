\section{Introduction}

%TAC - Trading Agent Competition is a strategy game which has clients, agents and services,it is a 5 day service and %includes fight tickets, hotel bookings and entertainment tickets. The motive of the agent is the maximise the profits by %buying and selling optimally.

The word TAC abbreviates Trading Agent Competition, its a bidding and auction game played between agent and the owner. The aim of the game is to provide best service to the clients depending on clients preferences. The theme of the game is, it's a five days holiday trip from TACTown to Tampa. The agent has to provide a package with three services. They are flight ticket from TACTown to Tampa, hotel room for shelter and entertainment for refreshment. Eight strategies are given by  eight agents, the winner of the game can be declared by who is having the best strategy.
  
                                 The agent buys flight tickets from TACAIR airline company for roundtrip through auctioning and bidding. The agent gets the ticket from the owner and sells it to his clients at optimum cost with no losses to the agent. As there are no return flights on the first day and last day ofthe trip the clients has to stay one night in Tampa, so hotel room is needed. Hotel rooms are of two types. They are luxury rooms and moderate rooms. Luxury rooms are bit costlier with many facilities where as moderate rooms are bit cheaper with less fecilities. Finally entertainment, there are three types of entertainment provided. They are amusement park, museum and alligator wrestling. Any of these tickets can be purchased by the clients according to their preference. Entertainment tickets are bought for higher cost from the owner and sold to the clients for low cost or for free, that depends on the agent strategy. 
 

                                     The agent has to give the best strategy in such a way that it is more feasible for the clients so that there may be a chance of buying more tickets from them. In our TAC game, we have given a strategy in the favour of client simultaneously with benfit to the agent without any loss.  

